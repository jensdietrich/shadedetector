\section{Evaluation}
\label{sec:evaluation}

\subsection{Dataset Selection}

The evaluation dataset consists of vulnerabilities and the associates artifacts in the Maven repository for which those vulnerabilities have been reported. The selection was driven by the following considerations: (1) to select widely used artifacts, as determined by the number of downstream clients reported by Maven~\footnote{\todo[inline]{add details}} (2) to select CVEs of different types, namely arbitrary code execution (AOE) and denial of service (DOS) (3) to include some high-impact vulnerabilities that have been exploited such as log4shell and the deserialisation vulnerability~\footnote{\todo[inline]{add details}} (4) to select libraries with a different purpose. 

Since our aim was to make CVEs testable in order to design a precise analysis, we furthermore gave preference to CVEs with available proof-of-concept (poc) projects we could then wrap or modify. In particular for vulnerabilties that have a high severity, such projects exist. Sometimes project covering entire classes of vulnerabilities can be used for this purpose, a good example is \textit{ysoserial}~\footnote{\url{https://github.com/frohoff/ysoserial}} that also  contains a poc for CVE-2015-7501 which we used in a modified, testable form. 


\begin{table}
	\begin{tabular}{|p{4.5cm}p{2cm}p{3.5cm}|}
		\hline
		artifact (gav) & short name & description  \\ 
		\hline
		commons-collections:\-commons-collections:\-3.2.1 & collect & data structure library  \\
		com.alibaba:\-fastjson:\-1.2.80 & fast-json & JSON parser  \\
		org.yaml:\-snakeyaml:\-1.250 & snakeyaml & YAML parser \\
		org.apache.logging.\-log4j:\-log4j-core:\-2.14.1 & log4j & logging   \\
		org.apache-extras.beanshell.\-bsh.\-2.0b5 & beanshell & scripting DSL \\ 
		com.google.guava:\-guava:\-11.0.1 & guava & utilities and data structures \\
		org.apache.commons:\-commons-text:\-1.9 & c-text & string utilities \\
		\hline
	\end{tabular}
	\caption{\label{tab:artifacts}Artifacts used in evaluation}
\end{table}





\begin{table}
	\begin{tabular}{|p{3cm}p{2cm}p{2.4cm}p{2.2cm}|}
		\hline
		CVE & artifact & CWEs  & severity  \\ 
		\hline
		CVE-2015-7501 & collect &  502  &  9.8 critical \\
		CVE-2016-2510 & beanshell & 19 & 8.1 high \\ 
		CVE-2018-10237  & guava & 770 & 5.9 medium  \\
	    CVE-2022-25845 & fastjson  & 502  & 9.8  critical \\
		CVE-2022-38749 & snakeyaml & 121, 787  &  6.5 medium \\
		CVE-2021-44228 & log4j & 20, 400, 502, 17 & 10  critical \\
		CVE-2022-42889 & c-text & 94 & 9.8 critical \\ 
		\hline
		
		\hline
	\end{tabular}
	\caption{\label{tab:cves}CVEs used in evaluation}
\end{table}




Table~\ref{tab:cves} lists the vulnerabilities we studied, cross-referenced with artifacts using the short names defined in Table~\ref{tab:artifacts}. Vulnerability meta data (CWEs and severities are sourced from the National Vulnerability Database (NVD), obtained from \url{https://nvd.nist.gov/vuln/detail/<CVE>} accessed on TODO.


\subsection{SCA Tool Selection}


There are numerous tools available to detect the presence of vulnerabilities through dependencies. 



\subsection{Tool Configuration}


\subsection{Results}